\documentclass[12pt]{report}
\setlength\parindent{0pt}
\setlength{\parskip}{1em}


\begin{document}
\chapter*{Week 1 Lecture 1}
\section*{Concepts}
Couloumb's Law is defined by the formula $F = \frac{\hat{r} k Q_1 Q_2}{r^2}$. In this case, $k \approx 9 * 10^9$. k is absolutely defined as $k = \frac{1}{4 \epsilon_{0}}$, but this is not relevant right now. $\hat{r}$ is defined as the unit vector from the other charge to the charge whose force you are calculating. $Q_1$ and $Q_2$ are the two charges in question.

Given that Couloumb's Law produces a vector, one may use the vector sum to find the net force on any given charge in a system of charges. 

\section*{Example Problem 1}
Given a system of three charges, $a$ located at the origin with a charge of $Q_a$, charge $b$ at $(3, 4)$ with charge $Q_b$, charge $c$ at $(x, 0)$. Find the value of $x$ so that the horizontal component of the force on $a$ is zero?




\end{document}
