 \documentclass[12pt]{report}
\setlength\parindent{0pt}
\setlength{\parskip}{1em}


\begin{document}
\chapter*{MATH 241 Lecture 1}
\section*{3d plane}
X axis out of page, y to the right, z upwards

3d Cartesian coordinate system

Rotate 2nd axis about 1st axis by 90 degrees counterclockwise to get 3rd axis

May use pythagorean theorem to get distance between two points in 3d cartesian space:

$$D = \sqrt{(x_1 - x_2)^2 + (y_1 - y_2)^2 + (z_1 - z_2)^2}$$

the 3d plane is the set of points $(x,y,z) where x,y,z \in \mathbb{R}$

\section*{Surface}
adding an equation reduces DOF by 1

In 2d plane, adding an equation goes from 2 DOF(plane) to 1 DOF(line)

Therefore, in 3d plane adding an equation goes from 3DOF(space) to 2dof(surface)

We can get 2d plane surface by the equation $z=0$

if an equation is unconstrained on an axis, it will extend along said axis

for example, the eqation $x^2 + y^2 = 4$ is a circle in the 2d space and a cylinder centered around the z axis in the 3d space

Sphere centered at $(h,k,l)$ can be described by $(x-h)^2 + (y-k)^2 + (z-l)^2 = r^2$


\end{document}
