\documentclass[12pt]{report}
\setlength\parindent{0pt}
\setlength{\parskip}{1em}


\begin{document}
\chapter*{MATH 241 Lecture 2}
\section*{Vectors in 3d} 

Vector addition is equivalent to drawing them back to back

Scalar multiplication by vector, direction stays the same, magnitude multiplied by scalar

\section*{Vector components} 

Vectors may be resolved into vectors on 1 plane, and operated on as you normally would

\section*{Standard Basis Vectors}

$$\hat{i} = <1,0,0>$$

$$\hat{j} = <0,1,0>$$

$$\hat{k} = <0,0,1>$$

All vectors can be rewritten as sums of scalar factors of standard basis vectors

Vectors can represent displacements and other motion
