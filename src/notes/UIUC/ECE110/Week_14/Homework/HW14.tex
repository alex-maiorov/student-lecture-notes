\documentclass[12pt]{report}
\setlength\parindent{0pt}
\setlength{\parskip}{1em}


\begin{document}
\chapter*{Homework 14}
Alexander Maiorov 660240339
\section*{Problem 1} 
We may establish the relationship between bandgap and wavelength by the formula $E_{p} = 1240/\lambda$. We may also establish the proportionality behaviour of photon flux by $\phi \propto \frac{P}{E_{p}}$. Therefore, $\phi \propto P * \lambda$. By that we can get:

$$\frac{r_1}{r_2} = \frac{P_1 * \lambda_1}{P_2 * \lambda_2}$$

We may define constants $a = 1mW$ and $b = 565nm$ to simplify our calculation. Therefore:

$$\frac{r_1}{r_2} = \frac{a * b}{4a * 4b} = \frac{1}{16} = 0.0625$$

\newpage
\section*{Problem 2}

We may calculate required power by multiplying $P = IV = 10 * 120 = 1200$. We may calculate the effective power per area by $\phi = \phi_{max} * Eff = $, where $\phi$ is measured in $W/m^2$. Therefore:

$$ A = \frac{IV}{\phi_{max} * Eff} = \frac{1200}{200} = 6m^2$$

Given the above, we may calculate daily generation in kWH by multiplying hourly power generated by the hours of sunlight $T$. We may find annual savings $S$ by multiplying that by 365 times the utility rate $R$. Therefore:

$$S = 365PTR = 365 * 1.2 * 7 * 0.38 = \$1165.08$$

I would like to thank the graders for grading our homework during this semester. 
\end{document}

 

